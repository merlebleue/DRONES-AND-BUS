Ce template centre les caption, place les \emph{figures} et les \emph{tables} là où elles se trouvent dans le code, propose une police et un style de \lstinline[language=Python]{code Python}. La largeur des images est de \lstinline{0.8\linewidth} par défaut. Des packages pratiques comme \emph{enumitem} et \emph{tabularray} sont aussi chargés. La police utilisée est \textbf{Linux Libertine O} pour le texte et \textbf{Libertinus} pour le mode math.

Pour donner quelques exemples\footnote{Un exemple}, d'abord quelques opérateurs dans le texte
$$\times$$
$\int\int\int\limits_{Q}f(x,y,z) \diff x \diff y \diff z$
puis
$\prod_{\gamma\in\Gamma_{\bar{C}}}\partial(\tilde{X}_\gamma)$ puis en mode math :
\[
  \int\int\int\int\limits_{Q}f(w,x,y,z) \diff w \diff x \diff y \diff z
  \leq
  \oint_{\partial Q} f'\left(\max\left\{
  \frac{\Vert w\Vert}{\vert w^2+x^2\vert};
  \frac{\Vert z\Vert}{\vert y^2+z^2\vert};
  \frac{\Vert w\oplus z\Vert}{\vert x\oplus y\vert}
  \right\}\right)\,
\]
\[
    \frac{1}{2\pi i} \int\limits_\gamma f\left(x^{\mathbf{N}\in\mathbb{C}^{N\times 10}}\right)
    = \sum_{k=1}^m n(\gamma;a_k)\mathrm{Res}(f;a_k)\,
  \]
  \[
    \max\{\, |f(z)|:z\in G^- \,\} = \max\{\, |f(z)|:z\in \partial G \,\}\,
  \]


Reccomandation: \emph{tabularray} est bien pratique pour faire de jolies tables:

\bigskip

\NewColumnType{A}{Q[l, yellow!50]}
\begin{tblr}{colspec={AQ[r,red!50]r}, hlines={2pt, gray}, vlines={2pt, gray}}
    \SetCell[c=3]{c, green, fg=red, font=\LARGE} StackExchange Sites && \\
    \SetRow{blue!50} Site & questions & answers \\
    Stack Overflow & 22m & 33m \\
    Mathematics & 1.5m & \SetCell{magenta} 2m \\
    Super User &  472k & 684k \\
    \SetRow{orange!50} TeX - LaTeX & 228k & 293k \\
\end{tblr}

\lstinputlisting[language=Python]{misc/somecode.py}